\documentclass[conference]{IEEEtran}
\IEEEoverridecommandlockouts
% The preceding line is only needed to identify funding in the first footnote. If that is unneeded, please comment it out.
\usepackage{cite}
\usepackage{amsmath,amssymb,amsfonts}
\usepackage{algorithmic}
\usepackage{graphicx}
\usepackage{textcomp}
\usepackage{xcolor}
\bibliographystyle{IEEEtran}

\begin{document}

\title{Conference Paper Title*\\
{\footnotesize \textsuperscript{*}Note: Sub-titles are not captured in Xplore and
should not be used}
}

\author{\IEEEauthorblockN{1\textsuperscript{st} André Lucena Ribas Ferreira}
    \IEEEauthorblockA{
        \textit{University of Minho}\\
        Braga, Portugal \\
        pg52672@uminho.pt}
    \and
    \IEEEauthorblockN{1\textsuperscript{st} Carlos Eduardo da Silva Machado}
    \IEEEauthorblockA{
        \textit{University of Minho}\\
        Braga, Portugal \\
        pg52675@uminho.pt}
    \and
    \IEEEauthorblockN{1\textsuperscript{st} Goncalo Manuel Maia de Sousa}
    \IEEEauthorblockA{
        \textit{University of Minho}\\
        Braga, Portugal \\
        pg52682@uminho.pt}
}

\maketitle

\begin{abstract}
    This document is a model and instructions for \LaTeX.
    This and the IEEEtran.cls file define the components of your paper [title, text, heads, etc.]. *CRITICAL: Do Not Use Symbols, Special Characters, Footnotes,
    or Math in Paper Title or Abstract.
\end{abstract}

\begin{IEEEkeywords}
    component, formatting, style, styling, insert
\end{IEEEkeywords}

\section{Introduction}
\begin{itemize}
    \item Machine learning has increased in popularity
    \begin{itemize}
        \item image classification
        \item natrual language proccessing
    \end{itemize}
    \item studies have tried to analyse I/O patterns in DL Workflows (source)
    \item very few get down to kernel level
    \item eBPF are ...
    \item we seek to provide a tool to Characterize DL workloads using eBPF's
\end{itemize}

\section{Background}

\begin{itemize}
    \item DL involves iterating multiple times (epochs) through a dataset
    \item all data is read exactly once one epoch as passed (I/O intensive)
    \item passing it through all the layers to calculate a loss (forward pass)
    \item use calculated loss to update the learnable parameters of the network (backpropagation)
    \item SGD is an optimizer for loss function minimization widely used for its lower computation  
    \item DL is usually I/O-bound [need source], due to the use of accelerators (GPU), size of the data and random reads
    \item pytorch is a DL framework
    \item Distributed DNN training (data paralellism)
    \item checkpointing involves saving the model state
    \item in pytorch its done explicity with torch.save() and in official workloads is done in-between epochs
    \item eBPF's
\end{itemize}

\section{Related work}

\begin{itemize}
    \item papers que usam darshan/tf-darshan para caracterizar padrões
    \item MLPerf Storage/tese de um aluno da Oana
    \item DIO e tools de observabilidade que usam eBPF e outras (related work do DIO), LD PRELOAD, captura de de I/O request por intrumentação do código fonte.
    \item Caracterizar Tensorflow I/O workloads através da sua definição teórica \cite{8638422}
    \item O que falta fazer?
    \begin{itemize}
        \item Análise empírica dos padrões como parte do processo de treino 
        \item Análise da cache como interveniente no processo de I/O
        \item Testes de Rede (para modelos distribuídos)
        \item Analisar PyTorch
    \end{itemize}
\end{itemize}

\section{Design}

\begin{itemize}
    \item Grafana
    \item python parser and plots
\end{itemize}

\section{Evaluation Methodology}

\begin{itemize}
    \item dstat, nvidia-smi to get cost of using the tool
    \item grafana dashboard to get data 
\end{itemize}

\section{Evaluation Results}

\section{Conclusion}

\bibliography{IEEEabrv, refs}

\end{document}
